\section{Introduction}

This is the text for my introduction. Do not forget that the paper should be not more than 7 pages long (it can go to 10 pages, but very few publications allow that. Do not forget that 7 pages include pictures and tables!

This would be a citation \cite{dummy}.

\ac{SN} 
% The first time you use this, the acronym will be written in full with the acronym in parentheses: supernova (SN). At later times it will just print the acronym: SN.

\acf{SN}
% written out form with acronym in parentheses, irrespective of previous use

\acs{SN}
% acronym form, irrespective of previous use

\acl{SN}
% written out form without following acronym

\acp{SN}
% plural form of acronym by adding an s. \acfp. \acsp, \aclp work as well.

As seen in \cite{wiki}.

\section{My section}

This is the text for my section. Do not forget that the paper should be not more than 7 pages long (it can go to 10 pages, but very few publications allow that). You can put pictures, table and equations, like you can see in source code.

\begin{figure}[H]
	\centering
		\includegraphics[width=0.5\linewidth]{dummy.pdf}
	\caption[Dummy Figure Caption for List of Figures.]{Dummy Figure Caption.}
	\label{fig:dummyfigure1}
\end{figure}

Remember you can change the reference style. Another dummy citation \cite{site}.

\begin{equation}
\dot{\mathbf{x}}(t) = \mathbf{T}\mathbf{z}(y),\  \mathbf{y}(0) = \mathbf{y}_0,\  z\geq 0 \\
\label{eq:dummyeq1}
\end{equation}

\noindent where

\begin{equation}
\mathbf{A} = \left[ \begin{array}{cc} -(a_{12} + a_{10}) & a_{21} \\ a_{12} & -(a_{21} + a_{20}) \end{array} \right]\\
\label{eq:dummyeq2}
\end{equation}

\begin{table}[H]
	\centering
	\caption{Dummy Table.}
	\begin{tabular}{|c|c|} \hline
		\textbf{Vendor Name} 				& \textbf{Short Name}	\\ \hline
		\multirow{3}{*}{Text in Multiple Row}		&	ABC			\\ \cline{2-2}
		 								&        DEF			\\ \cline{2-2}
										&        GHF	\\ \hline
		Text in Single Row					&        IJK			\\ \hline
		Frescos SA						&        LMN	\\ \hline
 \multicolumn{2}{|c|}{Text in Multiple Column}							\\ \hline
	\end{tabular}
	\label{tab:dummytable}
\end{table}

\section{Conclusion}

This is the text for your conclusions. It should be simple, as short as possible and state clearly the purpose and results of the thesis subject.

