\typeout{}
\typeout{--------------------------------------------------------------}
\typeout{ +---+ Article Tempate                             }
\typeout{ +---+      Version 2.0, August 2011                         }
\typeout{ +---+  for Instituto Superior Tecnico (IST),                 }
\typeout{ +---+  Universidade T�nica de Lisboa                         }
\typeout{ * Modified from J. Seixas original template                                }
\typeout{ * Created to write Master Thesis Articles                             }
\typeout{ * Conforms with IST Master Degree format                     }
\typeout{ AUTHOR: Miguel Amador                                          }
\typeout{--------------------------------------------------------------}
\typeout{}

%% Language %%%%%%%%%%%%%%%%%%%%%%%%%%%%%%%%%%%%%%%%%%%%%%%%%
\usepackage[portuguese,english]{babel} %francais, polish, spanish, ...
\usepackage[T1]{fontenc}
\usepackage[ansinew]{inputenc}
\usepackage{lmodern} %Type1-font for non-english texts and characters
\usepackage{float} %permite usar H (here) no posicionamento de floats

\raggedbottom

% correct bad hyphenation here
\hyphenation{op-tical net-works semi-conduc-tor}

% PACKAGE graphics, epsfig, subfigure, caption:
% ---------------------------------------------
% Packages for figures... well you will certainly need these packages, with the exception
% of the 'caption' package. This only allows to define extra caption options.
% Notice that subfigure allows to place figures within figures with its own caption. It
% should be avoided to create an eps file with subfigures. That will mean that you won't be 
% able to reference those subfigures. Instead create an EPS file (the only graphics format supported
% by latex) for each of the subfigures and then use the command \subfigure (see below).
\usepackage{graphics}
\usepackage{graphicx}
\usepackage{epsfig}
\usepackage[hang,small,bf]{subfigure}
%\usepackage{tikz} %%Generate vector graphics from within LaTeX
\usepackage{xcolor}
\usepackage{calc}
\usepackage{dcolumn}
\usepackage{bm}
\usepackage{booktabs}
\usepackage{rotating}
\usepackage[font=small,labelfont=bf,textfont=normalfont]{caption}
\usepackage{float}

\graphicspath{{Figures/}} %Definition of Figures Path

% PACKAGE latexsym:
% -----------------
% Defines additional latex symbols. May be required for thesis with many math symbols.
\usepackage{latexsym}

% PACKAGE amsmath, amsthm, amssymb, amsfonts:
% -------------------------------------------
% This package is typically required. Among many other things it adds the possibility
% to put symbols in bold by using \boldsymbol (not \mathbf); defines additional 
% fonts and symbols; adds the \eqref command for citing equations. I prefer the style
% "(x.xx)" for referering to an equation than to use "equation x.xx".
\usepackage{amsmath, amsthm, amssymb, amsfonts, amsbsy,amssymb}

% PACKAGE algorithmic, algorithm
% ------------------------------
% These packages are required if you need to describe an algorithm.
% \usepackage{algorithmic}
% \usepackage[chapter]{algorithm}

% PACKAGE natbib/cite
% -------------------
% The two packages are not compatible, and you should use one of the two. Notice however that the
% IEEE BiBTeX stylesheet is imcompatible with the natbib package. If using the IEEE format, use the 
% cite package instead
\usepackage[square,numbers,sort&compress]{natbib}
%\usepackage{cite}

% PACKAGE multirow, colortbl, longtable:
% ---------------------------------------
% These packages are most usefull for advanced tables. The first allows to join rows 
% throuhg the command \multirow which works similarly with the command \multicolumn
% The second package allows to color the table (both foreground and background)
% The third package is only required when tables extend beyond the length of one page;
% which typically does not happen and should be avoided
\usepackage{multirow}
\usepackage{multicol}
\usepackage{colortbl}
\usepackage{longtable} %%For tables, that exceed one page
\setlength\columnsep{20pt}

%% Line Spacing %%%%%%%%%%%%%%%%%%%%%%%%%%%%%%%%%%%%%%%%%%%%%
\renewcommand{\baselinestretch}{1.5}
\usepackage{setspace}
%\singlespacing        %% 1-spacing (default)
%\onehalfspacing       %% 1,5-spacing
%\doublespacing        %% 2-spacing


% PACKAGE acronyum
% -----------------
% This package is most useful for acronyms. The package guarantees that all acronyms definitions are 
% given at the first usage. IMPORTANT: do not use acronyms in titles/captions; otherwise the definition 
% will appear on the table of contents.
\usepackage[printonlyused]{acronym}
\usepackage[titletoc,title,header]{appendix}
\usepackage[noauto]{chappg}

%% Other Packages %%%%%%%%%%%%%%%%%%%%%%%%%%%%%%%%%%%%%%%%%%%
%\usepackage{a4wide} %%Smaller margins = more text per page.
\usepackage{fancyhdr} %%Fancy headings
\usepackage{geometry}
\usepackage{forarray}
\usepackage{array}
\usepackage{indentfirst}
\usepackage{url}
%\usepackage{hyperref}
\usepackage{verbatim}
% PACKAGE hyperref
% -----------------
% Set links for references and citations in document
% Some MiKTeX distributions have faulty PDF creators in which case this package will not work correctly
% Long live Linux :D
\usepackage[plainpages=false]{hyperref}


%% Defenitions %%%%%%%%%%%%%%%%%%%%%%%%%%%%%%%%%%%%%%%%%%%
%\textwidth 16cm
%\textheight 21cm
\geometry{top=3cm, bottom=2cm, left=2cm, right=2cm} %Defini��o das margens

%%%%%%%%%%%%%%%%%%%%%%%%%%%%%%%%%%%%%%%%%%%%%%%%%%%%%%%%%%%%%
%% Cabe�alhos
%%%%%%%%%%%%%%%%%%%%%%%%%%%%%%%%%%%%%%%%%%%%%%%%%%%%%%%%%%%%%
%\pagestyle{fancy} %Para introduzir cabe�alhos e rodap�s
%
%%Primeira P�gina
%
%\fancypagestyle{primeira.pag}{%
%\fancyhead{}
%\headheight 1.5cm
%\fancyhf{} % clear all header and footer fields
%\rhead{\footnotesize Subject\\ Course\\Y� Ano, X� Semestre 2010/2011} %Cabe�alho alinhado � direita (\\ para mudar de linha)
%\fancyfoot[C]{\large \thepage} % except the center
%\renewcommand{\headrulewidth}{0pt} %Sem linha no cabe�alho
%\renewcommand{\footrulewidth}{0.4pt}} %Com linha rodap�
%
%
%
%%Restantes p�ginas
%
%\fancypagestyle{outras.pag}{%
%\fancyhead{}
%\headheight 30pt
%\lhead{\normalsize Article Name}
%\rhead{\normalsize Authors} %Cabe�alho alinhado � direita
%\fancyfoot[C]{\large \thepage} % except the center
%\renewcommand{\footrulewidth}{0.4pt} %Linha do rodap�
%\renewcommand{\headrulewidth}{0.4pt}} %Linha do cabe�alho

%%%%%%%%%%%%%%%%%%%%%%%%%%%%%%%%%%%%%%%%%%%%%%%%%%%%%%%%%%%%%
%% User Functions
%%%%%%%%%%%%%%%%%%%%%%%%%%%%%%%%%%%%%%%%%%%%%%%%%%%%%%%%%%%%%
% Simple References
\newcommand{\figref}[1]{Figure \ref{#1}}
\newcommand{\equationref}[1]{Equation (\ref{#1})}
\newcommand{\tableref}[1]{Table (\ref{#1})}

%Integral
\newcommand{\infinityint}[1]{\int_{-\infty}^{\infty}#1}

%% Define a new 'leo' style for the package that will use a smaller font.
\makeatletter
\def\url@leostyle{%
  \@ifundefined{selectfont}{\def\UrlFont{\rm}}{\def\UrlFont{\large\rmfamily}}}
\makeatother
%% Now actually use the newly defined style.
\urlstyle{leo}